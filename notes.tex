\documentclass[fontset=windows]{article}
\usepackage{ctex}


\title{学习笔记}
\author{xj}
\date{\today}

\begin{document}

\maketitle

\section{GIT}
\begin{enumerate}
    \item git init //
    \item git add  //每次修改完都需要添加
    \item git commit -m "xxxx"
    \item git log //--pretty=oneline 减少输出信息
    \item git reset --hard HEAD\^{} //回退到上一个版本,HEAD\^{}\^{}上上个版本,HEAD\~{}100前100个版本
    \item git reset --hard [id]  //id为版本号
    \item git diff 
    \item git rm -r --cached . //撤销所有add

\end{enumerate}

\subsection{远程仓库}
\begin{enumerate}
    \item git remote add <name> git@server-name:path/repo-name.git //关联远程库
    \item git remote rm <name> //删除远程连接
    \item git remote -v //查看远程连接
    \item git push origin master //提交到远程库,第一次-u
\end{enumerate}

\subsection{分支}
\begin{enumerate}
    \item git switch -c dev //创建并切换到分支dev
    \item git switch master //切换到已有分支master
    \item git branch //查看分支
    \item git merge <name> //合并某分支到当前分支
    \item git branch -d <name> //删除分支
\end{enumerate}

\subsection{标签}
\begin{enumerate}
    \item git tag //查看tag
    \item git show <tag> //显示tag信息
    \item git tag -a <tag> -m "message" //创建tag
    \item git push origin <tag> //共享tag
    \item git tag -d <tag> //删除本地tag
    \item git push origin :refs/tags/<tag> //删除远程tag
    \item git fetch origin tag <tag> //拉取指定tag的版本
\end{enumerate}

\subsection{报错}
\begin{enumerate}
    \item git config core.autocrlf false // CRLF will be replaced by LF ...
\end{enumerate}


\section{\LaTeX}

\end{document}
